\documentclass[
   %draft,     % Entwurfsstadium
   final,      % fertiges Dokument
%%%% --- Schriftgröße ---
   10pt,
   %smallheadings,    % kleine Überschriften
   %normalheadings,   % normale Überschriften
   %bigheadings,       % große Überschriften
   headings=big,
%%%% --- Sprache ---
   ngerman,           % wird an andere Pakete weitergereicht
%%%% === Seitengröße ===
   % letterpaper,
   % legalpaper,
   % executivepaper,
   a4paper,
   % a5paper,
   % landscap,
%%%% === Optionen für den Satzspiegel ===
   BCOR5mm,          % Zusaetzlicher Rand auf der Innenseite
   DIV11,            % Seitengroesse (siehe Koma Skript Dokumentation !)
   %DIVcalc,         % automatische Berechnung einer guten Zeilenlaenge
   1.1headlines,     % Zeilenanzahl der Kopfzeilen
   headinclude=false,% Kopf einbeziehen / nicht einbeziehen
   footinclude=false,% Fuss einbeziehen / nicht einbeziehen
   mpinclude=false,  % Margin einbeziehen / nicht einbeziehen
   pagesize,         % Schreibt die Papiergroesse in die Datei.
                     % Wichtig fuer Konvertierungen
%%%% === Layout ===
   oneside,         % einseitiges Layout
   %twoside,          % Seitenraender für zweiseitiges Layout
   %onecolumn,        % Einspaltig
   %twocolumn,       % Zweispaltig
   %openany,         % Kapitel beginnen auf jeder Seite
   %openright,        % Kapitel beginnen immer auf der rechten Seite
                     % (macht nur bei 'twoside' Sinn)
   %cleardoubleplain,    % leere, linke Seite mit Seitenstil 'plain'
   %cleardoubleempty,    % leere, linke Seite mit Seitenstil 'empty'
   titlepage,        % Titel als einzelne Seite ('titlepage' Umgebung)
   %notitlepage,     % Titel in Seite integriert
%%%% --- Absatzeinzug ---
   %                 % Absatzabstand: Einzeilig,
   %parskip,         % Freiraum in letzter Zeile: 1em
   %parskip*,        % Freiraum in letzter Zeile: Viertel einer Zeile
   %parskip+,        % Freiraum in letzter Zeile: Drittel einer Zeile
   %parskip-,        % Freiraum in letzter Zeile: keine Vorkehrungen
   %                 % Absatzabstand: Halbzeilig
   %halfparskip,     % Freiraum in letzter Zeile: 1em
   %halfparskip*,    % Freiraum in letzter Zeile: Viertel einer Zeile
   %halfparskip+,    % Freiraum in letzter Zeile: Drittel einer Zeile
   %halfparskip,     % Freiraum in letzter Zeile: keine Vorkehrungen
   %                 % Absatzabstand: keiner
   %parindent,        % Eingerückt (Standard)
   parskip=true,
%%%% --- Kolumnentitel ---
	 abstracton,			 % Überschrift über den Abstract
   headsepline,      % Linie unter Kolumnentitel
   %headnosepline,   % keine Linie unter Kolumnentitel
   %footsepline,     % Linie unter Fussnote
   %footnosepline,   % keine Linie unter Fussnote
%%%% --- Kapitel ---
   chapterprefix,   % Ausgabe von 'Kapitel:'
   %nochapterprefix,  % keine Ausgabe von 'Kapitel:'
%%%% === Verzeichnisse (TOC, LOF, LOT, BIB) ===
   %listof=totoc,      % Tabellen & Abbildungsverzeichnis ins TOC
   %index=totoc,        % Index ins TOC
   %bibliography=totoc,         % Bibliographie ins TOC
   %bibtotocnumbered, % Bibliographie im TOC nummeriert
   %liststotocnumbered, % Alle Verzeichnisse im TOC nummeriert
   toc=graduated,        % eingereuckte Gliederung
   %tocleft,         % Tabellenartige TOC
   %listof=graduated,      % eingereuckte LOT, LOF
   %listsleft,       % Tabellenartige LOT, LOF
   %pointednumbers,  % Überschriftnummerierung mit Punkt, siehe DUDEN !
   numbers=noenddot, % Überschriftnummerierung ohne Punkt, siehe DUDEN !
   %openbib,         % alternative Formatierung des Literaturverzeichnisses
%%%% === Matheformeln ===
   %leqno,           % Formelnummern links
   fleqn,            % Formeln werden linksbuendig angezeigt
]{scrreprt}
\setlength{\parindent}{0cm} % Absätze nicht einrücken
\setlength{\parskip}{0.65em}
\usepackage[utf8]{inputenc}
\usepackage[ngerman]{babel}
%%% Links ===================================================
\usepackage[
	% Farben fuer die Links
	colorlinks=true,         % Links erhalten Farben statt Kaeten
	urlcolor=pdfurlcolor,    % \href{...}{...} external (URL)
	filecolor=pdffilecolor,  % \href{...} local file
	linkcolor=pdflinkcolor,  %\ref{...} and \pageref{...}
	citecolor=pdfcitecolor,
	% Links
	raiselinks=true,		% calculate real height of the link
	breaklinks,				% Links berstehen Zeilenumbruch
	%backref=page,			% Backlinks im Literaturverzeichnis (section, slide, page, none)
	%pagebackref=true,		% Backlinks im Literaturverzeichnis mit Seitenangabe
	verbose,
	hyperindex=true,		% backlinkex index
	linktocpage=true,		% Inhaltsverzeichnis verlinkt Seiten
	hyperfootnotes=false,	% Keine Links auf Fussnoten
	% Bookmarks
	bookmarks=true,			% Erzeugung von Bookmarks fuer PDF-Viewer
	bookmarksopenlevel=1,	% Gliederungstiefe der Bookmarks
	bookmarksopen=true,		% Expandierte Untermenues in Bookmarks
	bookmarksnumbered=true,	% Nummerierung der Bookmarks
	bookmarkstype=toc,		% Art der Verzeichnisses
	% Anchors
	plainpages=false,		% Anchors even on plain pages ?
	pageanchor=true,		% Pages are linkable
	pdfstartview=Fit,		% Dokument wird Fit Width geaefnet
	pdfpagemode=UseOutlines,	% Bookmarks im Viewer anzeigen
	pdfpagelabels=true,		% set PDF page labels
	pdfpagelayout=SinglePage
]{hyperref}
% ===========================================================

% Laden diverser Pakete======================================
\usepackage{booktabs}
\usepackage{colortbl}
\usepackage{url}
\usepackage{cite}
\usepackage{graphicx}
\usepackage{setspace}
\usepackage{tabularx}
\usepackage{multirow}
\usepackage{xcolor}
\usepackage{tocbasic}
\usepackage{titlesec}
\usepackage{lastpage} 
%\setlength{\textwidth}{14.75cm}
\setlength{\headsep}{1cm}
%\usepackage[top=2.5cm, bottom=4.5cm, headsep=1cm]{geometry}
% ===========================================================



%%% Verweise =============================================================
%
%%% Doc: Documentation inside dtx File
\usepackage[ngerman]{varioref} % Intelligente Querverweise

%-- Einstellungen für VarioRef --
\renewcommand*{\reftextfaceafter}{[siehe \reftextvario{nachfolgende}{kommende}{folgende}{nächste} Seite]} % Verweis auf die nächste Seite
\renewcommand*{\reftextfacebefore}{[siehe \reftextvario{vorherige}{vorhergehende} Seite]} % Verweis auf die vorherige Seite
\renewcommand*{\reftextafter}{\reftextfaceafter}
\renewcommand*{\reftextbefore}{\reftextfacebefore}
\renewcommand*{\reftextcurrent}{[siehe \reftextvario{diese}{aktuelle} Seite]} % Verweis auf derselben Seite
\renewcommand*{\reftextfaraway}[1]{[\reftextvario{siehe auch}{siehe} \reftextvario{Kap.}{Kapitel} \ref{#1} auf \reftextvario{Seite}{S.} \pageref{#1}]} % Verweis auf nicht angrenzende Seite
%\renewcommand*{\vref}[1]{\vpageref{#1}} % Alle Refs nur noch Pageref !!! ansonsten steht Kapitelnummer direkt vor dem eigentlichen VarioRef-Text!!!
\renewcommand*{\vrefrange}[2]{\vpagerefrange{#1}{#2}} % RefRange auf PageRefRange umbiegen, selbes Problem wie vref und vpageref
\newcommand{\vpicref}[1]{Abb. \ref{#1} \vref{#1}}
% ===========================================================



\setcounter{secnumdepth}{2}		% Abbildungsnummerierung mit groesserer Tiefe
\setcounter{tocdepth}{2}		% Inhaltsverzeichnis mit groesserer Tiefe



% Hauptpfade  ===============================================
\graphicspath{{Bilder/}}
\makeatletter
\def\input@path{{content/}}
\makeatother
% ===========================================================


% Kapitel + Nummer + Trennlinie + Name + Trennlinie  ========
\titleformat{\chapter}[display]	% {command}[shape]
  {\usekomafont{chapter}\huge \color{black}}	% format
  {												% label
  \LARGE\MakeUppercase{\chaptertitlename} \Huge \thechapter \filright%
  }%}
  {1pt}										% sep (from chapternumber)
  {\titlerule \vspace{0.9pc} \filright \color{chaptercolor}}    
  %{before}[after] (before chaptertitle and after)
  [\color{black} \vspace{0.9pc} \filright{\titlerule}]

\titleformat*{\section}{\color{sectioncolor}\Large\sffamily}
\titleformat*{\subsection}{\color{sectioncolor}\Large\sffamily}
\titleformat*{\subsubsection}{\color{sectioncolor}\large\sffamily}


%%% Doc: ftp://tug.ctan.org/pub/tex-archive/macros/latex/contrib/caption/caption.pdf
\usepackage{caption}
% Aussehen der Captions
\captionsetup{
   margin = 10pt,
   font = {small,rm},
   labelfont = {small,bf},
   format = plain, % oder 'hang'
   indention = 0em,  % Einruecken der Beschriftung
   labelsep = colon, %period, space, quad, newline
   justification = RaggedRight, % justified, centering
   singlelinecheck = true, % false (true=bei einer Zeile immer zentrieren)
   position = bottom %top
}
% ===========================================================


% Farben ====================================================
% Farbe der Ueberschriften
\definecolor{subsectioncolor}{RGB}{72, 118, 255} % Blau
\definecolor{chaptercolor}{RGB}{0, 0, 130} % Blau (dunkler))
\definecolor{sectioncolor}{RGB}{0, 0, 0}    % Schwarz
%
% Farbe des Textes
\definecolor{textcolor}{RGB}{0, 0, 0}    % Schwarz
%
% Farbe fuer grau hinterlegte Boxen (fuer Paket framed.sty)
\definecolor{shadecolor}{gray}{0.90}

% Farben fuer die Links im PDF
\definecolor{pdfurlcolor}{rgb}{0.6,0,0} % light red
\definecolor{pdffilecolor}{rgb}{0,0.6,0} % bright green
\definecolor{pdflinkcolor}{rgb}{0,0,0.75} % blue
\definecolor{pdfcitecolor}{rgb}{0,0,0} % black

% Farben fuer Listings
\colorlet{stringcolor}{green!40!black!100}
\colorlet{commencolor}{blue!0!black!100}
\definecolor{emphcolor}{rgb}{0,0,0.75}
\definecolor{lightgray}{rgb}{.9, .9, .9}
\definecolor{blue}{RGB}{58,95,205}
\definecolor{black}{RGB}{0, 0, 0}
\definecolor{darkgray}{rgb}{.4,.4,.4}
% ===========================================================


% Codelisting ===============================================
\usepackage{listings}
\lstset{
  numbers=left,               % Ort der Zeilennummern
  numberstyle=\tiny,          % Stil der Zeilennummern
  stepnumber=2,               % Abstand zwischen den Zeilennummern
  numbersep=-5pt,             % Abstand der Nummern zum Text
  tabsize=2,                  % Groesse von Tabs
  extendedchars=true,         %
  breaklines=true,            % Zeilen werden Umgebrochen
  stringstyle=\color{stringcolor}, % Farbe der String
	keywordstyle=\bfseries\color{violet},
	ndkeywordstyle=\bfseries\color{blue},
	commentstyle=\itshape\color{darkgray},
	emphstyle=\color{red},
	basicstyle=\small\ttfamily\color{black}\singlespacing, % Standardschrift
  showspaces=false,			% Leerzeichen anzeigen ?
  showtabs=false,			% Tabs anzeigen ?
  showstringspaces=false,	% Leerzeichen in Strings anzeigen ?
  frame=tlrb,				% Rahmen um das Listing (Großbuchstaben = doppelt)
  captionpos=b,						% Position der Caption (t|b)
	belowcaptionskip=\medskipamount,	%Platz unter der Caption
	aboveskip=\medskipamount,		% Platz über dem Listing
	backgroundcolor=\color{lightgray}, % Hintergrundfarbe der Box
	numberbychapter=true,			% Nummerierung der Listings nach den Kapiteln
	rulesepcolor=\color{white},		% Farbe des Hintergrunds zwischen innerem und äußerem Rahmen
	rulesep=0pt,						% Abstand der Rahmen voneinander
	%framexleftmargin=2mm,			% Box nach links erweitern
	firstnumber=1,					% Beginne mit 1 als Zeilennummer
	escapechar=\#
 }

\lstdefinelanguage{JavaScript}{
  keywords={typeof, new, true, false, catch, function, return, null, catch, switch, var, if, in, while, do, else, case, break},
  ndkeywords={class, export, boolean, throw, implements, import, this},
  sensitive=false,
  comment=[l]{//},
  morecomment=[s]{/*}{*/},
  morestring=[b]',
  morestring=[b]"
}

\lstdefinelanguage{Franca}{
  keywords={package, typeCollection, interface, attribute, is, method, version},
  ndkeywords={enumeration, typedef, struct, array, in, out, String, Float, Int8, Int16, Int32, Int64, UInt8, UInt16, UInt32, UInt64, major, minor},
  sensitive=false,
  comment=[l]{//},
  morecomment=[s]{<**}{**>},
  morestring=[b]',
  morestring=[b]"
}

\lstdefinelanguage{C++}{
  keywords={typeof, new, true, false, catch, function, return, null, catch, switch, var, if, in, while, do, else, case, break},
  ndkeywords={class, export, boolean, throw, implements, import, this, auto, int},
  sensitive=false,
  comment=[l]{//},
  morecomment=[s]{/*}{*/},
  morestring=[b]',
  morestring=[b]"
}

\lstdefinelanguage{XML}{
  keywords={DOCTYPE, PUBLIC, ELEMENT, ATTLIST, xml},
  alsoletter={:,-},
  morekeywords={xsl:stylesheet, xsl:template, xsl:applytemplates, xsl:value-of, xsl:apply-templates },
  ndkeywords={version, select, match},
  sensitive=false,
  comment=[s]{!--}{--},
  morecomment=[s]{<?}{?>},
  morestring=[s]{>}{<},
  morestring=[b]',
  morestring=[b]"
}
 
\lstloadlanguages{XML, HTML, C++, JavaScript}
% ===========================================================

%
\usepackage{fancyhdr}
%Einstelllungen Kopfzeile
\fancyhead[L]{}
%\renewcommand{\headrulewidth}{0.0pt} %obere Trennlinie entfernt

\setlength{\intextsep}{1.5\baselineskip} % Platz ober- und unterhalb des Bildes

%Einstellungen Fußzeile
\fancyfoot[L]{\textsf{Sascha Moser}}
\fancyfoot[C]{}
\fancyfoot[R]{\textsf{Seite {\thepage} von \pageref{LastPage}}}

\title{Test und Validierung von generiertem Middleware Sourcecode}
\author{Sascha Moser}
\date{15. September 2014}

\setkomafont{captionlabel}{\sffamily}

% Hurenkinder und Schusterjungen verhindern
\clubpenalty=10000
\widowpenalty=10000
\displaywidowpenalty=10000
%%%%%

\usepackage[german=quotes]{csquotes}